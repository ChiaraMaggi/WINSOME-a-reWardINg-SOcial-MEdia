\documentclass[11pt, a4paper, oneside]{article}
\usepackage[italian]{babel}
\usepackage[margin=3cm]{geometry}
\RequirePackage[latin1,utf8]{inputenc}
\usepackage[T1]{fontenc}
\usepackage[latin1]{inputenc}
\usepackage[bookmarks=true]{hyperref}
\usepackage[small]{titlesec}

\begin{document}
\title{Relazione Progetto Laboratorio di Reti: Winsome, a reWardINg SOcial MEdia}
\author{Università di Pisa - Dipartimento di Informatica \\ Chiara Maggi, 578517}
\date{A.A. 2021/22}
\maketitle
\tableofcontents

\section{Introduzione}
Il progetto realizzato rappresenta un social network con funzionalità base, dove ciascun utente registrato può seguire altri utenti ed essere a sua volta seguito. 
Questo meccanismo permette di presentare a un utente solo i contenuti pubblicati dagli utenti che egli segue. L’utente viene ricompensato dal servizio se pubblica
contenuti che riscuotono interesse da parte della comunità e/o se contribuisce attivamente votando o commentando contenuti pubblicati da altri utenti.\\
L'utente ha la possibilità di interagire con il social network da linea di comando, digitando particolari notazioni a seconda della richiesta,
sfruttando un processo client che sarà correttamente servito da uno dei thread del pool del server. Il server sfrutta la connessione TCP o RMI (a seconda della richiesta)
per eleaborare una risposta da mandare al client e quindi da mostrare all'utente.\\ \\
Una prima scelta implementativa è appunto quelle di realizzare il server con meccanismo Java I/O e Threadpool (in particolare CachedThreadpool) per andare a gestire
i vari client, così da poter identificare univocamente lo stato di login di un determinato utente e da gestire indipendentemente dagli altri le richieste di quest'ultimo.\\ \\
Un'altra scelta implementativa riguarda la gestione degli accessi di un utente al social network: in particolare un utente che ha fatto il login su un determinato 
client non potrà collegarsi su un client differente fino a quando non effetuerà il logout dal primo.\\ \\
Per quanto riguarda i sistemi di notifica ammessi dal client vi sono: RMI callback per ottenere le informazioni sui followers dell'utente collegato e servizio Multicast 
per ottenere le notifiche del periodico calcolo delle ricompense e conseguente aggiornamento del portafoglio. 















\end{document}